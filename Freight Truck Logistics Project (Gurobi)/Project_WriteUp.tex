% Options for packages loaded elsewhere
\PassOptionsToPackage{unicode}{hyperref}
\PassOptionsToPackage{hyphens}{url}
%
\documentclass[
]{article}
\usepackage{lmodern}
\usepackage{amssymb,amsmath}
\usepackage{ifxetex,ifluatex}
\ifnum 0\ifxetex 1\fi\ifluatex 1\fi=0 % if pdftex
  \usepackage[T1]{fontenc}
  \usepackage[utf8]{inputenc}
  \usepackage{textcomp} % provide euro and other symbols
\else % if luatex or xetex
  \usepackage{unicode-math}
  \defaultfontfeatures{Scale=MatchLowercase}
  \defaultfontfeatures[\rmfamily]{Ligatures=TeX,Scale=1}
\fi
% Use upquote if available, for straight quotes in verbatim environments
\IfFileExists{upquote.sty}{\usepackage{upquote}}{}
\IfFileExists{microtype.sty}{% use microtype if available
  \usepackage[]{microtype}
  \UseMicrotypeSet[protrusion]{basicmath} % disable protrusion for tt fonts
}{}
\makeatletter
\@ifundefined{KOMAClassName}{% if non-KOMA class
  \IfFileExists{parskip.sty}{%
    \usepackage{parskip}
  }{% else
    \setlength{\parindent}{0pt}
    \setlength{\parskip}{6pt plus 2pt minus 1pt}}
}{% if KOMA class
  \KOMAoptions{parskip=half}}
\makeatother
\usepackage{xcolor}
\IfFileExists{xurl.sty}{\usepackage{xurl}}{} % add URL line breaks if available
\IfFileExists{bookmark.sty}{\usepackage{bookmark}}{\usepackage{hyperref}}
\hypersetup{
  pdftitle={ISYE 3133 Semester Project Spring 2021},
  hidelinks,
  pdfcreator={LaTeX via pandoc}}
\urlstyle{same} % disable monospaced font for URLs
\usepackage[margin=1in]{geometry}
\usepackage{longtable,booktabs}
% Correct order of tables after \paragraph or \subparagraph
\usepackage{etoolbox}
\makeatletter
\patchcmd\longtable{\par}{\if@noskipsec\mbox{}\fi\par}{}{}
\makeatother
% Allow footnotes in longtable head/foot
\IfFileExists{footnotehyper.sty}{\usepackage{footnotehyper}}{\usepackage{footnote}}
\makesavenoteenv{longtable}
\usepackage{graphicx,grffile}
\makeatletter
\def\maxwidth{\ifdim\Gin@nat@width>\linewidth\linewidth\else\Gin@nat@width\fi}
\def\maxheight{\ifdim\Gin@nat@height>\textheight\textheight\else\Gin@nat@height\fi}
\makeatother
% Scale images if necessary, so that they will not overflow the page
% margins by default, and it is still possible to overwrite the defaults
% using explicit options in \includegraphics[width, height, ...]{}
\setkeys{Gin}{width=\maxwidth,height=\maxheight,keepaspectratio}
% Set default figure placement to htbp
\makeatletter
\def\fps@figure{htbp}
\makeatother
\setlength{\emergencystretch}{3em} % prevent overfull lines
\providecommand{\tightlist}{%
  \setlength{\itemsep}{0pt}\setlength{\parskip}{0pt}}
\setcounter{secnumdepth}{-\maxdimen} % remove section numbering

\title{ISYE 3133 Semester Project Spring 2021}
\author{}
\date{\vspace{-2.5em}}

\begin{document}
\maketitle

\textbf{Zach Ellison, Wyatt Entrekin, Rocco Russo}

\hypertarget{formulation}{%
\subsection{Formulation}\label{formulation}}

\[min \sum_{a\in A}t_am_a\]
{[}1{]}\[s.t. \sum_{a\in A_n^+}c_{a,k} -\sum_{a\in A_n^-}c_{a,k}=b_{k,n}\]
\[\forall n \in N, \forall k \in K\]
{[}2{]}\[\sum_{k\in K}(w_kc_{a,k})+e_a=t_a\] \[\forall a \in A\]
\[t_a \in \mathbb{Z}^+, c_{a,k} \in {0,1}, e_a \in [0,1]\]

\begin{longtable}[]{@{}ll@{}}
\toprule
\textbf{Choice Parameters} & \textbf{Definition}\tabularnewline
\midrule
\endhead
\(t_a\) & number of trucks on arc \emph{a}\tabularnewline
\(c_{a,k}\) & whether commodity \emph{k} is transported on arc
\emph{a}\tabularnewline
\(e_a\) & amount of empty space on last truck taking arc
\emph{a}\tabularnewline
\bottomrule
\end{longtable}

\begin{longtable}[]{@{}ll@{}}
\toprule
\textbf{Set} & \textbf{Definition}\tabularnewline
\midrule
\endhead
\emph{N} & All nodes in network\tabularnewline
\emph{A} & All arcs in network\tabularnewline
\emph{K} & All commodities\tabularnewline
\(A_n^+\) & All arcs ending at node \emph{n}\tabularnewline
\(A_n^-\) & All arcs beginning at node \emph{n}\tabularnewline
\bottomrule
\end{longtable}

\begin{longtable}[]{@{}ll@{}}
\toprule
\textbf{Other Variables} & \textbf{Definition}\tabularnewline
\midrule
\endhead
\(m_a\) & cost of using arc \emph{a} for one truck\tabularnewline
\(w_k\) & total weight of commodity \emph{k}\tabularnewline
\(o_k\) & source node for commodity \emph{k}\tabularnewline
\(d_k\) & sink node for commodity \emph{k}\tabularnewline
\bottomrule
\end{longtable}

\begin{longtable}[]{@{}ll@{}}
\toprule
\textbf{Demand Notation} & \textbf{Definition}\tabularnewline
\midrule
\endhead
\(b_{k,n}\) & demand of commodity \emph{k} at node \emph{n} such that
\(b_{k,n}\in {-1,1,0}\)\tabularnewline
\bottomrule
\end{longtable}

\begin{longtable}[]{@{}ll@{}}
\toprule
\textbf{Demand Value} & \textbf{Criteria}\tabularnewline
\midrule
\endhead
\(b_{k,n} = -1\) & \(n=o_k\), or the source node\tabularnewline
\(b_{k,n} = 1\) & \(n=t_k\), or the sink node\tabularnewline
\(b_{k,n}= 0\) & otherwise, or at any transit node\tabularnewline
\bottomrule
\end{longtable}

\hypertarget{formulation-explanations}{%
\subsection{Formulation Explanations}\label{formulation-explanations}}

\textbf{Objective Function}: The goal of the objective function is
minimizing cost of operation, which we have gathered from the the
problem description is defined as the sum over all arcs of the cost of
using that arc. The cost of using that arc is simply defined as the cost
of using a truck on a that arc times the number of trucks using that
arc.

\textbf{Constraints}:

\begin{enumerate}
\def\labelenumi{\arabic{enumi}.}
\tightlist
\item
  \textbf{Constraint 1} shown above is the conservation of flow
  constraint, which says that the difference in the flow into the node
  and out of the node per commodity, over all arcs, must equal the
  demand of commodity \emph{k} at node \emph{n}. We have defined the
  demand notation and constraints at the bottom, indicating that the
  only time in which flow is not conserved (demand is 0)are at the
  source node (demand is -1) and at the sink node (demand is 1)
\item
  \textbf{Constraint 2} shows that the weight of commodity \emph{k}
  times whether it is being transported on a certain arc \emph{a} summed
  over all commodities, added with the empty space on the last truck
  taking that arc equals the total number of trucks taking arc \emph{a}.
\end{enumerate}

\hypertarget{implementation---graph-parameter---explanations}{%
\subsection{Implementation - Graph Parameter -
Explanations}\label{implementation---graph-parameter---explanations}}

For \textbf{Commodities} and \textbf{Quantities}, we used a dictionary
where the keys are commodities and the values are quantities like the
hint says because whenever the code is dealing with a certain commodity,
it will be very easy to reference the quantity efficiently.

For \textbf{Nodes}, we used a dictionary where the key is the node (city
name) and the value is a tuple containing the x coordinate first and the
y coordinate second. Similarly to the Commodities and Quantities, the
coordinates will be very easy to reference.

For \textbf{Arcs}, we used a list of tuples where each tuple is (Head
Node, Tail Node). When iterating through the set of Arcs, it is very
easy to reference either the head or tail and refer to its coordinates
in the Nodes dictionary.

For \textbf{Sources}, we used a dictionary where the keys are
commodities and the values are sources. Like the Commodities and
Quantities dictionary, it will be very easy to reference the source of a
given commodity.

For \textbf{Sinks}, we used a dictionary where the keys are commodities
and the values are sinks. This follows the same reasoning as Sources.

For \textbf{Distances}, we used a dictionary where the keys are arcs in
tuple form (hence making Arcs a list of tuples) and the value is the
Euclidean distance calculated from the coordinate tuples.

The code for graph\_parameter.py is commented to show where we did the
following:

\begin{enumerate}
\def\labelenumi{\arabic{enumi}.}
\tightlist
\item
  Used Pandas to read the CSV files into dataframes.
\item
  Iterated through the rows of the appropriate dataframes to extract the
  specific data we needed in each of our data structures.
\item
  Iterated through our Arcs list to calculate distances.
\end{enumerate}

\hypertarget{implementation---gurobi-modeling---explanations}{%
\subsection{Implementation - Gurobi Modeling -
Explanations}\label{implementation---gurobi-modeling---explanations}}

In the function \textbf{get\_data()}, we use \textbf{graph\_parameter()}
to read in our data, create a few more data structures (a dictionary for
demand at each node for each commodity, a dictionary telling us how many
trucks are on each arc, and a list of commodites). We also, set up the
demand at each source and sink node to be ready for the model to run.

In the function \textbf{build\_model()}, we created the model, we
created our decision variables, then added our constraints, and lastly
set our objective function. To see more specific explanations,
\textbf{ModelA.py} is well commented and explains which specific lines
create specific parts of our Mixed-Integer Program formulation.

In \textbf{optimize\_model()}, we optimize the MIP and output the
results.

In the main part of the program, we wrote some code to format the
results and optimal solution into a readable format.

\hypertarget{implementation---solving---explanations}{%
\subsection{Implementation - Solving -
Explanations}\label{implementation---solving---explanations}}

We found an objective value solution of \$431.9. The runtime of the
program was 0.019274 seconds. The optimal solution which describes the
path of each commodity and the number of trucks on each arc can be found
in \textbf{ModelAOutput.txt}.

\hypertarget{analysis---path-a---explanations}{%
\subsection{Analysis - Path A -
Explanations}\label{analysis---path-a---explanations}}

The code for this analysis can be found in \textbf{ModelAPathA.py}. The
output for the optimal solution can be found in
\textbf{ModelAPathAOutput.txt}.

After changing the path-making decision variable to be continuous
instead of binary, while the number of trucks remained continuous, we
found that the objective function value did not change. The runtime was
similarly unaffected with both versions of the program running in around
0.01 - 0.09 seconds. The optimal solution remains the same as each
commodity takes the cheapest possible path from source to sink. The
paths and the number of trucks on each arc do not change from the
previous solution.

This makes sense as with a fractional number of trucks there is no
incentive to deviate from the optimal path to save money by trying to
fully load a truck. As we will further explore in Analysis Path B, the
amount of a commodity transported on a path is equal to the number of
trucks and as such the path decision making will be identical as there
is no incentive to send anything down anything but the most efficient
path because with fractional trucks, we are essentially sending the
commodity and there is no limit on weight or anything of the sort.
Ultimately there is no concern for sending partially empty trucks as
instead we are simply sending partial trucks.

\hypertarget{analysis---path-b---explanations}{%
\subsection{Analysis - Path B -
Explanations}\label{analysis---path-b---explanations}}

The number of trucks on each arc is equal to the weight of the commodity
being transported on the arc. As such we can rewrite the objective
function as \(min\sum_{a∈A}\sum_{k∈K}m_a w_k c_{a,k}\). This is the
minimization of the summation of the cost of sending a truck or the cost
of sending a weight of 1 on the arc times the weight of commodity k
times the binary variable determining whether k is sent on arc a. Thus
our updated formulation is as follows:

\[min\sum_{a∈A}\sum_{k∈K}m_a w_k c_{a,k}\]
{[}1{]}\[s.t. \sum_{a\in A_n^+}c_{a,k} -\sum_{a\in A_n^-}c_{a,k}=b_{k,n}\]
\[\forall n \in N, \forall k \in K\]
{[}2{]}\[\sum_{k\in K}(w_kc_{a,k})=t_a\] \[\forall a \in A\]
\[t_a \in \mathbb{Z}^+, c_{a,k} \in {0,1}, e_a \in [0,1]\]

\begin{longtable}[]{@{}ll@{}}
\toprule
\textbf{Choice Parameters} & \textbf{Definition}\tabularnewline
\midrule
\endhead
\(t_a\) & number of trucks on arc \emph{a}\tabularnewline
\(c_{a,k}\) & whether commodity \emph{k} is transported on arc
\emph{a}\tabularnewline
\bottomrule
\end{longtable}

\begin{longtable}[]{@{}ll@{}}
\toprule
\textbf{Set} & \textbf{Definition}\tabularnewline
\midrule
\endhead
\emph{N} & All nodes in network\tabularnewline
\emph{A} & All arcs in network\tabularnewline
\emph{K} & All commodities\tabularnewline
\(A_n^+\) & All arcs ending at node \emph{n}\tabularnewline
\(A_n^-\) & All arcs beginning at node \emph{n}\tabularnewline
\bottomrule
\end{longtable}

\begin{longtable}[]{@{}ll@{}}
\toprule
\textbf{Other Variables} & \textbf{Definition}\tabularnewline
\midrule
\endhead
\(m_a\) & cost of using arc \emph{a} for one truck\tabularnewline
\(w_k\) & total weight of commodity \emph{k}\tabularnewline
\(o_k\) & source node for commodity \emph{k}\tabularnewline
\(d_k\) & sink node for commodity \emph{k}\tabularnewline
\bottomrule
\end{longtable}

\begin{longtable}[]{@{}ll@{}}
\toprule
\textbf{Demand Notation} & \textbf{Definition}\tabularnewline
\midrule
\endhead
\(b_{k,n}\) & demand of commodity \emph{k} at node \emph{n} such that
\(b_{k,n}\in {-1,1,0}\)\tabularnewline
\bottomrule
\end{longtable}

\begin{longtable}[]{@{}ll@{}}
\toprule
\textbf{Demand Value} & \textbf{Criteria}\tabularnewline
\midrule
\endhead
\(b_{k,n} = -1\) & \(n=o_k\), or the source node\tabularnewline
\(b_{k,n} = 1\) & \(n=t_k\), or the sink node\tabularnewline
\(b_{k,n}= 0\) & otherwise, or at any transit node\tabularnewline
\bottomrule
\end{longtable}

This updated formulation is fairly similar to the formulation for the
transportation networks with a objective function defined by cost of
transportation times the amount transported with the constraints
focusing on satisfying supply and demand. Similarly this model only has
one decision variable the decision to transport commodity k on arc a.

\hypertarget{analysis---truck---explanations}{%
\subsection{Analysis - Truck -
Explanations}\label{analysis---truck---explanations}}

The code for this analysis can be found in \textbf{TruckAnalysis.py}.
The output for the optimal solution can be found in
\textbf{TruckAnalysisOutput.txt}.

The objective value of the solution is now 547.07 which is much larger
than the previous objective value. The runtime is also significantly
longer taking around 30 seconds instead of only taking less than a
second. The increase in runtime and the worsened optimization of the
solution is unsurprising as we are now using integer programming instead
of linear programming as a result the computer is having to do the
simplex method many more times than it did before we had integer trucks.
The solution is unsurprisingly less efficient as we now have empty
spaces on trucks and additional restrictions that take us farther from
the non integer original solution. These empty spaces mean we have to
use more trucks which cost more money and the computer also has to
calculate far more routes. Many commodities used the same route over
both programs however some changed their route by joining routes where
other commodities have extra space on their trucks allowing these
commodities to travel more efficiently as now instead of only taking 0.1
of a truck they need a full truck whose wasted space would be wasted
money. Unlike before the number of trucks on each arc does not simply
match the weight shipped on each arc and instead is the lowest integer
value greater than the weight shipped on the arc.

\hypertarget{extra-credit}{%
\subsection{Extra Credit}\label{extra-credit}}

\end{document}
